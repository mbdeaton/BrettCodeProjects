%%%%%%%%%%%%%%%%%%%%%%%%%%%%%%%%%%%%%%%%%%%%%%%%%%%%%%%%%%%%%%%%%
% Contents: Typesetting Part of LaTeX2e Introduction
% $Id: typeset.tex 423 2010-06-21 21:04:21Z oetiker $
%%%%%%%%%%%%%%%%%%%%%%%%%%%%%%%%%%%%%%%%%%%%%%%%%%%%%%%%%%%%%%%%%
\chapter{Typesetting Text}

\begin{intro}
  After reading the previous chapter, you should know about the basic
  stuff of which a \LaTeXe{} document is made. In this chapter I
  will fill in the remaining structure you will need to know in order
  to produce real world material.
\end{intro}

\section{The Structure of Text and Language}
\secby{Hanspeter Schmid}{hanspi@schmid-werren.ch}
The main point of writing a text (some modern DAAC\footnote{Different
  At All Cost, a translation of the Swiss German UVA (Um's Verrecken
  Anders).} literature excluded), is to convey ideas, information, or
knowledge to the reader.  The reader will understand the text better
if these ideas are well-structured, and will see and feel this
structure much better if the typographical form reflects the logical
and semantical structure of the content.

\LaTeX{} is different from other typesetting systems in that you just
have to tell it the logical and semantical structure of a text.  It
then derives the typographical form of the text according to the
``rules'' given in the document class file and in various style files.

The most important text unit in \LaTeX{} (and in typography) is the
\wi{paragraph}.  We call it ``text unit'' because a paragraph is the
typographical form that should reflect one coherent thought, or one idea. 
You will learn in the following sections how to force line breaks with
e.g.\ \texttt{\bs\bs}, and paragraph breaks with e.g.\ leaving an empty line
in the source code.  Therefore, if a new thought begins, a new paragraph
should begin, and if not, only line breaks should be used.  If in doubt
about paragraph breaks, think about your text as a conveyor of ideas and
thoughts.  If you have a paragraph break, but the old thought continues, it
should be removed.  If some totally new line of thought occurs in the same
paragraph, then it should be broken.

Most people completely underestimate the importance of well-placed
paragraph breaks.  Many people do not even know what the meaning of
a paragraph break is, or, especially in \LaTeX, introduce paragraph
breaks without knowing it.  The latter mistake is especially easy to
make if equations are used in the text.  Look at the following
examples, and figure out why sometimes empty lines (paragraph breaks)
are used before and after the equation, and sometimes not.  (If you
don't yet understand all commands well enough to understand these
examples, please read this and the following chapter, and then read
this section again.)

\begin{code}
\begin{verbatim}
% Example 1
\ldots when Einstein introduced his formula 
\begin{equation} 
  e = m \cdot c^2 \; , 
\end{equation} 
which is at the same time the most widely known 
and the least well understood physical formula. 


% Example 2
\ldots from which follows Kirchhoff's current law:
\begin{equation} 
  \sum_{k=1}^{n} I_k = 0 \; .
\end{equation} 

Kirchhoff's voltage law can be derived \ldots


% Example 3
\ldots which has several advantages.

\begin{equation} 
  I_D = I_F - I_R
\end{equation} 
is the core of a very different transistor model. \ldots
\end{verbatim}
\end{code} 

The next smaller text unit is a sentence.  In English texts, there is
a larger space after a period that ends a sentence than after one
that ends an abbreviation.  \LaTeX{} tries to figure out which one
you wanted to have.  If \LaTeX{} gets it wrong, you must tell it what
you want.  This is explained later in this chapter.

The structuring of text even extends to parts of sentences.  Most
languages have very complicated punctuation rules, but in many
languages (including German and English), you will get almost every
comma right if you remember what it represents: a short stop in the
flow of language.  If you are not sure about where to put a comma,
read the sentence aloud and take a short breath at every comma.  If
this feels awkward at some place, delete that comma; if you feel the
urge to breathe (or make a short stop) at some other place, insert a
comma.

Finally, the paragraphs of a text should also be structured logically
at a higher level, by putting them into chapters, sections,
subsections, and so on.  However, the typographical effect of writing
e.g.\ \verb|\section{The| \texttt{Structure of Text and Language}\verb|}| is
so obvious that it is almost self-evident how these high-level
structures should be used.

\section{Line Breaking and Page Breaking}
 
\subsection{Justified Paragraphs}

Books are often typeset with each line having the same length.
\LaTeX{} inserts the necessary \wi{line break}s and spaces between words
by optimizing the contents of a whole paragraph. If necessary, it
also hyphenates words that would not fit comfortably on a line.
How the paragraphs are typeset depends on the document class.
Normally the first line of a paragraph is indented, and there is no
additional space between two paragraphs. Refer to section~\ref{parsp}
for more information.

In special cases it might be necessary to order \LaTeX{} to break a
line: 
\begin{lscommand}
\ci{\bs} or \ci{newline} 
\end{lscommand}
\noindent starts a new line without starting a new paragraph. 

\begin{lscommand}
\ci{\bs*}
\end{lscommand}
\noindent additionally prohibits a page break after the forced
line break. 

\begin{lscommand}
\ci{newpage}
\end{lscommand}
\noindent starts a new page. 

\begin{lscommand}
\ci{linebreak}\verb|[|\emph{n}\verb|]|,
\ci{nolinebreak}\verb|[|\emph{n}\verb|]|, 
\ci{pagebreak}\verb|[|\emph{n}\verb|]|,
\ci{nopagebreak}\verb|[|\emph{n}\verb|]|
\end{lscommand}
\noindent suggest places where a break may (or may not happen). They enable the author to influence their
actions with the optional argument \emph{n}, which can be set to a number
between zero and four. By setting \emph{n} to a value below 4, you leave
\LaTeX{} the option of ignoring your command if the result would look very
bad. Do not confuse these ``break'' commands with the ``new'' commands. Even
when you give a ``break'' command, \LaTeX{} still tries to even out the
right border of the line and the total length of the page, as described in
the next section; this can lead to unpleasant gaps in your text.
If you really want to start a ``new line'' or a ``new page'', then use the
corresponding command. Guess their names!


\LaTeX{} always tries to produce the best line breaks possible. If it
cannot find a way to break the lines in a manner that meets its high
standards, it lets one line stick out on the right of the paragraph.
\LaTeX{} then complains (``\wi{overfull hbox}'') while processing the
input file. This happens most often when \LaTeX{} cannot find a
suitable place to hyphenate a word.\footnote{Although \LaTeX{} gives
  you a warning when that happens (\texttt{Overfull \bs{}hbox}) and displays the
  offending line, such lines are not always easy to find. If you use
  the option \texttt{draft} in the \ci{documentclass} command, these
  lines will be marked with a thick black line on the right margin.}
Instruct \LaTeX{} to lower its standards a little by giving
the \ci{sloppy} command. It prevents such over-long lines by
increasing the inter-word spacing---even if the final output is not
optimal.  In this case a warning (``\wi{underfull hbox}'') is given to
the user.  In most such cases the result doesn't look very good. The
command \ci{fussy} brings \LaTeX{} back to its default behaviour.

\subsection{Hyphenation} \label{hyph}

\LaTeX{} hyphenates words whenever necessary. If the hyphenation
algorithm does not find the correct hyphenation points,
remedy the situation by using the following commands to tell \TeX{}
about the exception.

The command
\begin{lscommand}
\ci{hyphenation}\verb|{|\emph{word list}\verb|}|
\end{lscommand}
\noindent causes the words listed in the argument to be hyphenated only at
the points marked by ``\verb|-|''.  The argument of the command should only
contain words built from normal letters, or rather signs that are considered
to be normal letters by \LaTeX{}. The hyphenation hints are
stored for the language that is active when the hyphenation command
occurs. This means that if you place a hyphenation command into the preamble
of your document it will influence the English language hyphenation. If you
place the command after the \verb|\begin{document}| and you are using some
package for national language support like \pai{babel}, then the hyphenation
hints will be active in the language activated through \pai{babel}.

The example below will allow ``hyphenation'' to be hyphenated as well as
``Hyphenation'', and it prevents ``FORTRAN'', ``Fortran'' and ``fortran''
from being hyphenated at all.  No special characters or symbols are allowed
in the argument.

Example:
\begin{code}
\verb|\hyphenation{FORTRAN Hy-phen-a-tion}|
\end{code}

The command \ci{-} inserts a discretionary hyphen into a word. This
also becomes the only point hyphenation is allowed in this word. This
command is especially useful for words containing special characters
(e.g.\ accented characters), because \LaTeX{} does not automatically
hyphenate words containing special characters.
%\footnote{Unless you are using the new
%\wi{DC fonts}.}.

\begin{example}
I think this is: su\-per\-cal\-%
i\-frag\-i\-lis\-tic\-ex\-pi\-%
al\-i\-do\-cious
\end{example}

Several words can be kept together on one line with the command
\begin{lscommand}
\ci{mbox}\verb|{|\emph{text}\verb|}|
\end{lscommand}
\noindent It causes its argument to be kept together under all circumstances.

\begin{example}
My phone number will change soon.
It will be \mbox{0116 291 2319}.

The parameter 
\mbox{\emph{filename}} should 
contain the name of the file.
\end{example}

\ci{fbox} is similar to \ci{mbox}, but in addition there will
be a visible box drawn around the content.


\section{Ready-Made Strings}

In some of the examples on the previous pages, you have seen
some very simple \LaTeX{} commands for typesetting special
text strings:

\vspace{2ex}

\noindent
\begin{tabular}{@{}lll@{}}
Command&Example&Description\\
\hline
\ci{today} & \today   & Current date\\
\ci{TeX} & \TeX       & Your favorite typesetter\\
\ci{LaTeX} & \LaTeX   & The Name of the Game\\
\ci{LaTeXe} & \LaTeXe & The current incarnation\\
\end{tabular}

\section{Special Characters and Symbols}
 
\subsection{Quotation Marks}

You should \emph{not} use the \verb|"| for \wi{quotation marks}
\index{""@\texttt{""}} as you would on a typewriter.  In publishing
there are special opening and closing quotation marks.  In \LaTeX{},
use two~\textasciigrave~(grave accent) for opening quotation marks and
two~\textquotesingle~(vertical quote) for closing quotation marks. For single
quotes you use just one of each.
\begin{example}
``Please press the `x' key.''
\end{example}
Yes I know the rendering is not ideal, it's really a back-tick or grave accent
(\textasciigrave) for
opening quotes and vertical quote (\textquotesingle) for closing, despite what the font chosen might suggest.

\subsection{Dashes and Hyphens}

\LaTeX{} knows four kinds of \wi{dash}es. Access three of
them with different number of consecutive dashes. The fourth sign
is actually not a dash at all---it is the mathematical minus sign: \index{-}
\index{--} \index{---} \index{-@$-$} \index{mathematical!minus}

\begin{example}
daughter-in-law, X-rated\\
pages 13--67\\
yes---or no? \\
$0$, $1$ and $-1$
\end{example}
The names for these dashes are: 
`-' \wi{hyphen}, `--' \wi{en-dash}, `---' \wi{em-dash} and
`$-$' \wi{minus sign}.

\subsection{Tilde ($\sim$)}
\index{www}\index{URL}\index{tilde}
A character often seen in web addresses is the tilde. To generate
this in \LaTeX{} use \verb|\~| but the result: \~{} is not really
what you want. Try this instead:

\begin{example}
http://www.rich.edu/\~{}bush \\
http://www.clever.edu/$\sim$demo
\end{example}  
 
\subsection{Degree Symbol \texorpdfstring{($\circ$)}{}}

Printing the \wi{degree symbol} in pure \LaTeX{}.

\begin{example}
It's $-30\,^{\circ}\mathrm{C}$.
I will soon start to
super-conduct.
\end{example}

The \pai{textcomp} package makes the degree symbol also available as
\ci{textdegree} or in combination with the C by using the \ci{textcelsius}.

\begin{example}
30 \textcelsius{} is
86 \textdegree{}F.
\end{example}

\subsection{The Euro Currency Symbol \texorpdfstring{(\officialeuro)}{}}

When writing about money these days, you need the Euro symbol. Many current
fonts contain a Euro symbol. After loading the \pai{textcomp} package in the preamble of your document
\begin{lscommand}
\ci{usepackage}\verb|{textcomp}| 
\end{lscommand}
use the command
\begin{lscommand}
\ci{texteuro}
\end{lscommand}
to access it.

If your font does not provide its own Euro symbol or if you do not like the
font's Euro symbol, you have two more choices:

First the \pai{eurosym} package. It provides the official Euro symbol:
\begin{lscommand}
\ci{usepackage}\verb|[|\emph{official}\verb|]{eurosym}|
\end{lscommand}
If you prefer a Euro symbol that matches your font, use the option
\texttt{gen} in place of the \texttt{official} option.

%If the Adobe Eurofonts are installed on your system (they are available for
%free from \url{ftp://ftp.adobe.com/pub/adobe/type/win/all}) you can use
%either the package \pai{europs} and the command \ci{EUR} (for a Euro symbol
%that matches the current font).
% does not work
% or the package
% \pai{eurosans} and the command \ci{euro} (for the ``official Euro'').

%The \pai{marvosym} package also provides many different symbols, including a
%Euro, under the name \ci{EURtm}. Its disadvantage is that it does not provide
%slanted and bold variants of the Euro symbol.

\begin{table}[!htbp]
\caption{A bag full of Euro symbols} \label{eurosymb}
\begin{lined}{10cm}
\begin{tabular}{llccc}
LM+textcomp  &\verb+\texteuro+ & \huge\texteuro &\huge\sffamily\texteuro
                                                &\huge\ttfamily\texteuro\\
eurosym      &\verb+\euro+ & \huge\officialeuro &\huge\sffamily\officialeuro
                                                &\huge\ttfamily\officialeuro\\
$[$gen$]$eurosym &\verb+\euro+ & \huge\geneuro  &\huge\sffamily\geneuro
                                                &\huge\ttfamily\geneuro\\
%europs       &\verb+\EUR + & \huge\EURtm        &\huge\EURhv
%                                                &\huge\EURcr\\
%eurosans     &\verb+\euro+ & \huge\EUROSANS  &\huge\sffamily\EUROSANS
%                                             & \huge\ttfamily\EUROSANS \\
%marvosym     &\verb+\EURtm+  & \huge\mvchr101  &\huge\mvchr101
%                                               &\huge\mvchr101
\end{tabular}
\medskip
\end{lined}
\end{table}

\subsection{Ellipsis (\texorpdfstring{\ldots}{...})}

On a typewriter, a \wi{comma} or a \wi{period} takes the same amount of
space as any other letter. In book printing, these characters occupy
only a little space and are set very close to the preceding letter.
Therefore, entering `\wi{ellipsis}' by just typing three
dots would be wrong produce the wrong result. Instead, there is a special
command for these dots. It is called

\begin{lscommand}
\ci{ldots}
\end{lscommand}
\index{...@\ldots}


\begin{example}
Not like this ... but like this:\\
New York, Tokyo, Budapest, \ldots
\end{example}
 
\subsection{Ligatures}

Some letter combinations are typeset not just by setting the
different letters one after the other, but by actually using special
symbols.
\begin{code}
{\large ff fi fl ffi\ldots}\quad
instead of\quad {\large f{}f f{}i f{}l f{}f{}i \ldots}
\end{code}
These so-called \wi{ligature}s can be prohibited by inserting an \ci{mbox}\verb|{}|
between the two letters in question. This might be necessary with
words built from two words.

\begin{example}
\Large Not shelfful\\
but shelf\mbox{}ful
\end{example}
 
\subsection{Accents and Special Characters}
 
\LaTeX{} supports the use of \wi{accent}s and \wi{special character}s
from many languages. Table~\ref{accents} shows all sorts of accents
being applied to the letter o. Naturally other letters work too.

To place an accent on top of an i or a j, its dots have to be
removed. This is accomplished by typing \verb|\i| and \verb|\j|.

\begin{example}
H\^otel, na\"\i ve, \'el\`eve,\\ 
sm\o rrebr\o d, !`Se\~norita!,\\
Sch\"onbrunner Schlo\ss{} 
Stra\ss e
\end{example}

\begin{table}[!hbp]
\caption{Accents and Special Characters.} \label{accents}
\begin{lined}{10cm}
\begin{tabular}{*4{cl}}
\A{\`o} & \A{\'o} & \A{\^o} & \A{\~o} \\
\A{\=o} & \A{\.o} & \A{\"o} & \B{\c}{c}\\[6pt]
\B{\u}{o} & \B{\v}{o} & \B{\H}{o} & \B{\c}{o} \\
\B{\d}{o} & \B{\b}{o} & \B{\t}{oo} \\[6pt]
\A{\oe}  &  \A{\OE} & \A{\ae} & \A{\AE} \\
\A{\aa} &  \A{\AA} \\[6pt]
\A{\o}  & \A{\O} & \A{\l} & \A{\L} \\
\A{\i}  & \A{\j} & !` & \verb|!`| & ?` & \verb|?`| 
\end{tabular}
\index{dotless \i{} and \j}\index{Scandinavian letters}
\index{ae@\ae}\index{umlaut}\index{grave}\index{acute}
\index{oe@\oe}\index{aa@\aa}

\bigskip
\end{lined}
\end{table}

\section{International Language Support}
\index{international} When you write documents in \wi{language}s
other than English, there are three areas where \LaTeX{} has to be
configured appropriately:

\begin{enumerate}
\item All automatically generated text strings\footnote{Table of
    Contents, List of Figures, \ldots} have to be adapted to the new
  language.  For many languages, these changes can be accomplished by
  using the \pai{babel} package by Johannes Braams.
\item \LaTeX{} needs to know the hyphenation rules for the new
  language. Getting hyphenation rules into \LaTeX{} is a bit more
  tricky. It means rebuilding the format file with different
  hyphenation patterns enabled. Your \guide{} should give more
  information on this.
\item Language specific typographic rules. In French for example, there is a
  mandatory space before each colon character (:).
\end{enumerate}

If your system is already configured appropriately, activate
the \pai{babel} package by adding the command
\begin{lscommand}
\ci{usepackage}\verb|[|\emph{language}\verb|]{babel}|
\end{lscommand}
\noindent after the \verb|\documentclass| command. A list of the
\emph{language}s built into your \LaTeX{} system will be displayed
every time the compiler is started. Babel will
automatically activate the appropriate hyphenation rules for the
language you choose. If your \LaTeX{} format does not support
hyphenation in the language of your choice, babel will still work but
will disable hyphenation, which has quite a negative effect on the
appearance of the typeset document.

\textsf{Babel} also specifies new commands for some languages, which
simplify the input of special characters. The \wi{German} language, for
example, contains a lot of umlauts (\"a\"o\"u).  With \textsf{babel} loaded,
enter an \"o by typing \verb|"o| instead of~\verb|\"o|.

If you call babel with multiple languages
\begin{lscommand}
\ci{usepackage}\verb|[|\emph{languageA}\verb|,|\emph{languageB}\verb|]{babel}| 
\end{lscommand}
\noindent then the last language in the option list will be active (i.e.
languageB). Use the command
\begin{lscommand}
\ci{selectlanguage}\verb|{|\emph{languageA}\verb|}|
\end{lscommand}
\noindent to change the active language.

%Input Encoding
\newcommand{\ieih}[1]{%
\index{encodings!input!#1@\texttt{#1}}%
\index{input encodings!#1@\texttt{#1}}%
\index{#1@\texttt{#1}}}
\newcommand{\iei}[1]{%
\ieih{#1}\texttt{#1}}
%Font Encoding
\newcommand{\feih}[1]{%
\index{encodings!font!#1@\texttt{#1}}%
\index{font encodings!#1@\texttt{#1}}%
\index{#1@\texttt{#1}}}
\newcommand{\fei}[1]{%
\feih{#1}\texttt{#1}}

Most of the modern computer systems allow you to input letter of
national alphabets  directly from the keyboard. In order to 
handle variety of input encoding used for different groups of 
languages and/or on different computer platforms \LaTeX{} employs the
\pai{inputenc} package:
\begin{lscommand}
\ci{usepackage}\verb|[|\emph{encoding}\verb|]{inputenc}|
\end{lscommand}

When using this package, you should consider that other people might not
be able to display your input files on their computer, because they use
a different encoding. For example, the German umlaut \"a on OS/2 is
encoded as 132, on Unix systems using ISO-LATIN~1 it is encoded as 228,
while in Cyrillic encoding cp1251 for Windows this letter does not exist
at all; therefore you should use this feature with care. The following
encodings may come in handy, depending on the type of system you are
working on\footnote{To learn more about supported  input
encodings for Latin-based and Cyrillic-based languages, read the
documentation for \texttt{inputenc.dtx} and \texttt{cyinpenc.dtx}
respectively. Section~\ref{sec:Packages} tells how to produce package
documentation.}

\begin{center}
\begin{tabular}{l | r | r }
Operating & \multicolumn{2}{c}{encodings}\\
system  & western Latin      & Cyrillic\\
\hline
Mac     &  \iei{applemac} & \iei{macukr}  \\
Unix    &  \iei{latin1}   & \iei{koi8-ru}  \\ 
Windows &  \iei{ansinew}  & \iei{cp1251}    \\
DOS, OS/2  &  \iei{cp850} & \iei{cp866nav}
\end{tabular}                
\end{center}                 

If you have a multilingual document with conflicting input encodings,
you might want to switch to unicode, using the \pai{ucs} package.


\begin{lscommand}
\ci{usepackage}\verb|{ucs}|\\ 
\ci{usepackage}\verb|[|\iei{utf8x}\verb|]{inputenc}| 
\end{lscommand}
\noindent will enable you to create \LaTeX{} input files in 
\iei{utf8x}, a multi-byte encoding in which each character can be encoded in
as little as one byte and as many as four bytes. 

Font encoding is a different matter. It defines at which position inside
a \TeX-font each letter is stored. Multiple input encodings could be mapped into 
one font encoding, which reduces number of required font sets.
Font encodings are handled through 
\pai{fontenc} package: \label{fontenc}
\begin{lscommand}
\ci{usepackage}\verb|[|\emph{encoding}\verb|]{fontenc}| \index{font encodings}
\end{lscommand}
\noindent where \emph{encoding} is font encoding. It is possible to load several
encodings simultaneously.

The default \LaTeX{} font encoding is \label{OT1} \fei{OT1}, the encoding of the
original Computer Modern \TeX{} font. It contains only the 128
characters of the 7-bit ASCII character set. When accented characters
are required, \TeX{} creates them by combining a normal character with
an accent. While the resulting output looks perfect, this approach stops
the automatic hyphenation from working inside words containing accented
characters. Besides, some of Latin letters could not be created by
combining a normal character with an accent, to say nothing about letters of
non-Latin alphabets, such as Greek or Cyrillic.

To overcome these shortcomings, several 8-bit CM-like font sets were created.
\emph{Extended Cork} (EC) fonts in \fei{T1} encoding contains 
letters and punctuation characters for most of the European
languages based on Latin script. The LH font set contains letters necessary
to typeset documents in languages using Cyrillic script. Because of the large
number of Cyrillic glyphs, they are arranged into four font
encodings---\fei{T2A}, \fei{T2B}, \fei{T2C},
and~\fei{X2}.\footnote{The list of languages supported by each of these
encodings could be found in \cite{cyrguide}.} The CB bundle contains fonts
in \fei{LGR} encoding for the composition of Greek text.

Improve/enable hyphenation in non-English
documents by using these fonts. Another advantage of using new CM-like fonts is that they 
provide fonts of CM families in all weights, shapes, and optically
scaled font sizes. 

\subsection{Support for Portuguese}

\secby{Demerson Andre Polli}{polli@linux.ime.usp.br}
To enable hyphenation and change all automatic text to \wi{Portuguese},
\index{Portugu\^es} use the command:
\begin{lscommand}
\verb|\usepackage[portuguese]{babel}|
\end{lscommand}
Or if you are in Brazil, substitute the language for \texttt{\wi{brazilian}}.

As there are a lot of accents in Portuguese you might want to use
\begin{lscommand}
\verb|\usepackage[latin1]{inputenc}|
\end{lscommand}
to be able to input them correctly as well as 
\begin{lscommand}
\verb|\usepackage[T1]{fontenc}|
\end{lscommand}
to get the hyphenation right.

See table~\ref{portuguese} for the preamble you need to write in the
Portuguese language. Note that we are using the latin1 input encoding here,
so this will not work on a Mac or on DOS. Just use
the appropriate encoding for your system.

\begin{table}[btp]
\caption{Preamble for Portuguese documents.} \label{portuguese}
\begin{lined}{5cm}
\begin{verbatim}
\usepackage[portuguese]{babel}
\usepackage[latin1]{inputenc}
\usepackage[T1]{fontenc}
\end{verbatim}
\bigskip
\end{lined}
\end{table}


\subsection{Support for French}

\secby{Daniel Flipo}{daniel.flipo@univ-lille1.fr}
Some hints for those creating \wi{French} documents with \LaTeX{}:
load French language support with the following command:

\begin{lscommand}
\verb|\usepackage[francais]{babel}|
\end{lscommand}

This enables French hyphenation, if you have configured your
\LaTeX{} system accordingly. It also changes all automatic text into
French: \verb+\chapter+ prints Chapitre, \verb+\today+ prints the current
date in French and so on. A set of new commands also
becomes available, which allows you to write French input files more
easily. Check out table \ref{cmd-french} for inspiration. 

\begin{table}[!htbp]
\caption{Special commands for French.} \label{cmd-french}
\begin{lined}{9cm}
\selectlanguage{french}
\begin{tabular}{ll}
\verb+\og guillemets \fg{}+         \quad &\og guillemets \fg \\[1ex]
\verb+M\up{me}, D\up{r}+            \quad &M\up{me}, D\up{r}  \\[1ex]
\verb+1\ier{}, 1\iere{}, 1\ieres{}+ \quad &1\ier{}, 1\iere{}, 1\ieres{}\\[1ex]
\verb+2\ieme{} 4\iemes{}+           \quad &2\ieme{} 4\iemes{}\\[1ex]
\verb+\No 1, \no 2+                 \quad &\No 1, \no 2   \\[1ex]
\verb+20~\degres C, 45\degres+      \quad &20~\degres C, 45\degres \\[1ex]
\verb+\bsc{M. Durand}+              \quad &\bsc{M.~Durand} \\[1ex]
\verb+\nombre{1234,56789}+          \quad &\nombre{1234,56789}
\end{tabular}
\selectlanguage{english}
\bigskip
\end{lined}
\end{table}

You will also notice that the layout of lists changes when switching to the
French language. For more information on what the \texttt{francais}
option of \textsf{babel} does and how to customize its behaviour, run
\LaTeX{} on file \texttt{frenchb.dtx} and read the produced file
\texttt{frenchb.dvi}.

Recent versions of \pai{frenchb} rely on \pai{numprint} to implement the \ci{nombre} command.

\subsection{Support for German}

Some hints for those creating \wi{German}\index{Deutsch}
documents with \LaTeX{}: load German language support with the following
command:

\begin{lscommand}
\verb|\usepackage[german]{babel}|
\end{lscommand}

This enables German hyphenation, if you have configured your
\LaTeX{} system accordingly. It also changes all automatic text into
German. Eg. ``Chapter'' becomes ``Kapitel.'' A set of new commands also
becomes available, which allows you to write German input files more quickly
even when you don't use the inputenc package. Check out table
\ref{german} for inspiration. With inputenc, all this becomes moot, but your 
text also is locked in a particular encoding world.

\begin{table}[!htbp]
\caption{German Special Characters.} \label{german}
\begin{lined}{8cm}
\selectlanguage{german}
\begin{tabular}{*2{ll}}
\verb|"a| & "a \hspace*{1ex} & \verb|"s| & "s \\[1ex]
\verb|"`| & "` & \verb|"'| & "' \\[1ex]
\verb|"<| or \ci{flqq} & "<  & \verb|">| or \ci{frqq} & "> \\[1ex]
\ci{flq} & \flq & \ci{frq} & \frq \\[1ex]
\ci{dq} & " \\
\end{tabular}
\selectlanguage{english}
\bigskip
\end{lined}
\end{table}

In German books you often find French quotation marks (\flqq guil\-le\-mets\frqq).
German typesetters, however, use them differently. A quote in a German book
would look like \frqq this\flqq. In the German speaking part of Switzerland,
typesetters use \flqq guillemets\frqq~the same way the French do.

A major problem arises from the use of commands
like \verb+\flq+: If you use the OT1 font (which is the default font) the  
guillemets will look like the math symbol ``$\ll$'', which turns a typesetter's stomach.
T1 encoded fonts, on the other hand, do contain the required symbols. So if you are using this type
of quote, make sure you use the T1 encoding. (\verb|\usepackage[T1]{fontenc}|)

\subsection[Support for Korean]{Support for Korean\footnotemark}\label{support_korean}%
\footnotetext{%
Considering a number of issues  Korean \LaTeX{} users
have to cope with.
This section was written by Karnes KIM on behalf of the
Korean lshort translation team. It  was translated into English
by SHIN Jungshik and shortened by Tobi Oetiker.}

To use \LaTeX{} for typesetting  \wi{Korean}, 
we need to solve three problems: 

\begin{enumerate}
\item 
We must be able to 
edit \wi{Korean input files}.
Korean input files must be in plain text format, but because Korean
uses its own character set outside the
repertoire of US-ASCII, they will look rather strange with a normal ASCII editor.  The two most widely used encodings for
Korean text files are  EUC-KR and its upward compatible
extension used in Korean MS-Windows, CP949/Windows-949/UHC.
In these encodings each US-ASCII character represents its normal ASCII
character similar to other ASCII compatible encodings such as
ISO-8859-\textit{x}, EUC-JP, Big5, or Shift\_JIS. On the other hand, Hangul
syllables, Hanjas (Chinese characters as used in Korea), Hangul Jamos,
Hiraganas, Katakanas, Greek and Cyrillic characters and other
symbols and letters drawn from KS~X~1001 are represented by two
consecutive octets. The first has its MSB set.
Until the mid-1990's, it took a considerable amount of time and effort to
set up a Korean-capable environment under a non-localized (non-Korean)
operating system. 
Skim through the now much-outdated \url{http://jshin.net/faq} to get 
a glimpse of what it was like to use Korean under non-Korean OS in mid-1990's.
These days all three major operating systems (Mac OS, Unix, Windows) come equipped
with pretty decent multilingual support and internationalization features
so that editing Korean text file is not so much of a problem anymore, even
on non-Korean operating systems.

\item \TeX{} and \LaTeX{} were originally written for
scripts with no more than 256 characters in their alphabet.
To make them work for languages with considerably 
more characters such as
Korean%,
 \footnote{Korean Hangul is an alphabetic script with 14 basic consonants
 and 10 basic vowels (Jamos). Unlike Latin or Cyrillic scripts, the
 individual characters have to be arranged in rectangular
 clusters about the same size as Chinese characters. Each cluster
 represents a syllable. An unlimited number of syllables can be
 formed out of this finite set of vowels and consonants. Modern Korean
 orthographic standards (both in South Korea and  North Korea), however,
 put some restriction on the formation of these clusters.
 Therefore only a finite number of  orthographically correct syllables exist.
 The Korean Character encoding defines individual code points for each of these syllables (KS~X~1001:1998 and KS~X~1002:1992). So Hangul, albeit alphabetic, is
 treated like the Chinese and Japanese writing systems with tens of thousands of
 ideographic/logographic characters.  ISO~10646/Unicode offers both ways of
 representing Hangul used for \emph{modern} Korean by encoding Conjoining
 Hangul Jamos (alphabets: \url{http://www.unicode.org/charts/PDF/U1100.pdf})
 in addition to encoding all the orthographically allowed Hangul syllables in
 \emph{modern} Korean (\url{http://www.unicode.org/charts/PDF/UAC00.pdf}).
 One of the most daunting challenges in Korean typesetting with
 \LaTeX{} and related typesetting system is supporting Middle Korean---and possibly future Korean---syllables that can be only represented
 by conjoining Jamos in Unicode. It is hoped that future \TeX{} engines like $\Omega$ and
 $\Lambda$ will eventually provide solutions to this
 so that some Korean linguists and historians
 will defect from MS Word that already has  a pretty good support 
 for Middle Korean.}
or Chinese, a subfont mechanism was developed.
It divides a single CJK font with  thousands or tens of thousands of
glyphs into a set of subfonts with 256 glyphs each. 
For Korean, there are three widely used packages;  \wi{H\LaTeX}
by UN~Koaunghi, \wi{h\LaTeX{}p} by CHA~Jaechoon and the \wi{CJK package}
by Werner~Lemberg.\footnote{% 
They can be obtained at \CTANref|language/korean/HLaTeX/|\\
   \CTANref|language/korean/CJK/| and
   \texttt{http://knot.kaist.ac.kr/htex/}}
H\LaTeX{} and h\LaTeX{}p are specific to Korean and provide
Korean localization on top of the font support.
They both can process Korean input text files encoded in EUC-KR. H\LaTeX{} can
even process input files encoded in CP949/Windows-949/UHC and UTF-8
when used along with $\Lambda$, $\Omega$.

The CJK package is not specific to Korean. It can
process input files in UTF-8 as well as in various CJK encodings
including EUC-KR and CP949/Windows-949/UHC, it can be used to typeset documents with
multilingual content (especially Chinese, Japanese and Korean).
The CJK package has no Korean localization such as the one offered by H\LaTeX{} and it
does not come with as many special Korean fonts as H\LaTeX.

\item The ultimate purpose of using typesetting programs like \TeX{}
and \LaTeX{} is to get documents typeset in an `aesthetically' satisfying way.
Arguably the most important element in typesetting is  a set of
well-designed fonts. The H\LaTeX{} distribution
includes \index{Korean font!UHC font}UHC \PSi{} fonts 
of 10
different families and
Munhwabu\footnote{Korean Ministry of Culture.}
fonts (TrueType) of 5 different families.
The CJK package works with a set of fonts used by earlier versions
of H\LaTeX{} and it can use Bitstream's cyberbit TrueType
font.
\end{enumerate}

To use the  H\LaTeX{} package for typesetting your Korean text, put the following
declaration into the preamble of your document:
\begin{lscommand}
\verb+\usepackage{hangul}+
\end{lscommand}

This command turns the Korean localization on. The headings
of chapters, sections, subsections, table of content and table of
figures are all translated into Korean and the formatting of the document
is changed to follow Korean conventions. 
The package also provides automatic ``particle selection.''
In Korean, there are pairs of post-fix particles 
grammatically equivalent but different in form. Which 
of any given pair is correct depends on 
whether the preceding syllable ends with a  vowel or a consonant.
(It is a bit more complex than this, but this should give you
a good picture.)
Native Korean speakers have no problem picking the right particle, but
it cannot be determined which particle to use for references and other automatic
text that will change while you edit the document.
It 
takes a painstaking effort to place appropriate particles manually
every time you add/remove references or simply shuffle  parts
of your document around.
H\LaTeX{} relieves its users from this boring and error-prone process.

In case you don't need Korean localization features
but just want 
to  typeset Korean text, put the following line in the 
preamble, instead.
\begin{lscommand}
\verb+\usepackage{hfont}+
\end{lscommand}

For more details on typesetting  Korean with H\LaTeX{}, refer to
the \emph{H\LaTeX{} Guide}.  Check out the web site of the Korean
\TeX{} User Group (KTUG) at  \url{http://www.ktug.or.kr/}.
There is also a Korean translation
of this manual available.

\subsection{Writing in Greek}
\secby{Nikolaos Pothitos}{pothitos@di.uoa.gr}
See table~\ref{preamble-greek} for the preamble you need to write in the
\wi{Greek} \index{Greek} language.  This preamble enables hyphenation and
changes all automatic text to Greek.\footnote{If you select the
\texttt{utf8x}
option for the package \texttt{inputenc}, \LaTeX{} will understand Greek and polytonic
Greek
unicode characters.}

\begin{table}[btp]
\caption{Preamble for Greek documents.} \label{preamble-greek}
\begin{lined}{7cm}
\begin{verbatim}
\usepackage[english,greek]{babel}
\usepackage[iso-8859-7]{inputenc}
\end{verbatim}
\bigskip
\end{lined}
\end{table}

A set of new commands also becomes available, which allows you to write
Greek input files more easily.  In order to temporarily switch to English
and vice versa, one can use the commands \verb|\textlatin{|\emph{english
text}\verb|}| and \verb|\textgreek{|\emph{greek text}\verb|}| that both take
one argument which is then typeset using the requested font encoding. 
Otherwise use the command \verb|\selectlanguage{...}| described in a
previous section.  Check out table~\ref{sym-greek} for some Greek
punctuation characters.  Use \verb|\euro| for the Euro symbol.

\begin{table}[!htbp]
\caption{Greek Special Characters.} \label{sym-greek}
\begin{lined}{4cm}
\selectlanguage{french}
\begin{tabular}{*2{ll}}
\verb|;| \hspace*{1ex}  &  $\cdot$ \hspace*{1ex}  &  \verb|?| \hspace*{1ex}&  ;   \\[1ex]
\verb|((|               &  \og                    &  \verb|))|&  \fg \\[1ex]
\verb|``|               &  `                      &  \verb|''| &  '   \\
\end{tabular}
\selectlanguage{english}
\bigskip
\end{lined}
\end{table}


\subsection{Support for Cyrillic}

\secby{Maksym Polyakov}{polyama@myrealbox.com}
Version~3.7h of \pai{babel} includes support for the
\fei{T2*}~encodings and for typesetting Bulgarian, Russian and
Ukrainian texts using Cyrillic letters.  

Support for Cyrillic is based on standard \LaTeX{} mechanisms plus
the \pai{fontenc} and \pai{inputenc} packages. But, if you are going to
use Cyrillics in math mode, you need to load \pai{mathtext} package
before \pai{fontenc}:\footnote{If you use \AmS-\LaTeX{} packages, 
load them before \pai{fontenc} and \pai{babel} as well.}
\begin{lscommand}
\verb+\usepackage{mathtext}+\\
\verb+\usepackage[+\fei{T1}\verb+,+\fei{T2A}\verb+]{fontenc}+\\
\verb+\usepackage[+\iei{koi8-ru}\verb+]{inputenc}+\\
\verb+\usepackage[english,bulgarian,russian,ukranian]{babel}+
\end{lscommand}

Generally, \pai{babel} will authomatically choose the default font encoding,
for the above three languages this is \fei{T2A}.  However, documents are not
restricted to a single font encoding. For multi-lingual documents using
Cyrillic and Latin-based languages it makes sense to include Latin font
encoding explicitly. \pai{babel} will take care of switching to the appropriate
font encoding when a different language is selected within the document.

In addition to enabling hyphenations, translating automatically
generated text strings, and activating some language specific 
typographic rules (like \ci{frenchspacing}), \pai{babel} provides some 
commands allowing typesetting according to the standards of 
Bulgarian, Russian, or Ukrainian languages. 


For all three languages, language specific punctuation is provided:
The Cyrillic dash for the text (it is little narrower than Latin dash and
surrounded by tiny spaces), a dash for direct speech, quotes, and
commands to facilitate hyphenation, see Table~\ref{Cyrillic}.

% Table borrowed from Ukrainian.dtx
\begin{table}[htb]
  \begin{center}
  \index{""-@\texttt{""}\texttt{-}} 
  \index{""---@\texttt{""}\texttt{-}\texttt{-}\texttt{-}} 
  \index{""=@\texttt{""}\texttt{=}} 
  \index{""`@\texttt{""}\texttt{`}} 
  \index{""'@\texttt{""}\texttt{'}} 
  \index{"">@\texttt{""}\texttt{>}} 
  \index{""<@\texttt{""}\texttt{<}} 
  \caption[Bulgarian, Russian, and Ukrainian]{The extra definitions made
           by Bulgarian, Russian, and Ukrainian options of \pai{babel}}\label{Cyrillic}
  \begin{tabular}{@{}p{.1\hsize}@{}p{.9\hsize}@{}}
   \hline
   \verb="|= & disable ligature at this position.               \\
   \verb|"-| & an explicit hyphen sign, allowing hyphenation
               in the rest of the word.                         \\
   \verb|"---| & Cyrillic emdash in plain text.                      \\
   \verb|"--~| & Cyrillic emdash in compound names (surnames).       \\
   \verb|"--*| & Cyrillic emdash for denoting direct speech.         \\
   \verb|""| & like \verb|"-|, but producing no hyphen sign
               (for compound words with hyphen, e.g.\ \verb|x-""y|
               or some other signs  as ``disable/enable'').     \\
   \verb|"~| & for a compound word mark without a breakpoint.        \\
   \verb|"=| & for a compound word mark with a breakpoint, allowing
          hyphenation in the composing words.                   \\
   \verb|",| & thinspace for initials with a breakpoint
           in following surname.                                \\
   \verb|"`| & for German left double quotes
               (looks like ,\kern-0.08em,).                     \\
   \verb|"'| & for German right double quotes (looks like ``).       \\%''
   \verb|"<| & for French left double quotes (looks like $<\!\!<$).  \\
   \verb|">| & for French right double quotes (looks like $>\!\!>$). \\
   \hline
  \end{tabular}
  \end{center}
\end{table}


The Russian and Ukrainian options of \pai{babel} define the commands \ci{Asbuk}
and \ci{asbuk}, which act like \ci{Alph} and \ci{alph}, but produce capital
and small letters of Russian or Ukrainian alphabets (whichever is the
active language of the document). The Bulgarian option of \pai{babel} 
provides the commands \ci{enumBul} and \ci{enumLat} (\ci{enumEng}), which
make \ci{Alph} and \ci{alph} produce letters of either
Bulgarian or Latin (English) alphabets. The default behaviour of
\ci{Alph}  and \ci{alph} for the Bulgarian language option is to
produce letters from the Bulgarian alphabet. 

%Finally, math alphabets are redefined and  as well as the commands for math
%operators according to Cyrillic typesetting traditions. 

\subsection{Support for Mongolian}

To use \LaTeX{} for typesetting Mongolian you have a choice between two packages:
Multilingual Babel and Mon\TeX{} by Oliver Corff.

Mon\TeX{} includes support for both Cyrillic and traditional
Mongolian Script. In order to access the commands of Mon\TeX{}, add:
\begin{lscommand}
\ci{usepackage}\verb|[|\emph{language},\emph{encoding}\verb|]{mls}|
\end{lscommand}
\noindent to the preamble. Choose the \emph{language} option \pai{xalx} to generate
captions and the dates in Modern Mongolian. To write a complete document in the traditional Mongolian script
you have to choose \pai{bicig} for the \emph{language} option. The document
language option \pai{bicig} enables the ``Simplified Transliteration'' input method.

Enable and disable Latin Transliteration Mode with
\begin{lscommand}
\verb|\SetDocumentEncodingLMC|
\end{lscommand}
and
\begin{lscommand}
\verb|\SetDocumentEncodingNeutral|
\end{lscommand}

More information about Mon\TeX{} is available from
\CTAN|language/mongolian/montex/doc|.

Mongolian Cyrillic script is supported by \pai{babel}. Activate
Mongolian language support with the following commands:

\begin{lscommand}
\verb|\usepackage[T2A]{fontenc}|\\
\verb|\usepackage[mn]{inputenc}|\\
\verb|\usepackage[mongolian]{babel}|
\end{lscommand}

\noindent where \iei{mn} is the \iei{cp1251} input encoding. For a more modern approach
invoke \iei{utf8} instead.

\section{The Space Between Words}

To get a straight right margin in the output, \LaTeX{} inserts varying
amounts of space between the words. It inserts slightly more space at
the end of a sentence, as this makes the text more readable.  \LaTeX{}
assumes that sentences end with periods, question marks or exclamation
marks. If a period follows an uppercase letter, this is not taken as a
sentence ending, since periods after uppercase letters normally occur in
abbreviations.

Any exception from these assumptions has to be specified by the
author. A backslash in front of a space generates a space that will
not be enlarged. A tilde~`\verb|~|' character generates a space that cannot be
enlarged and additionally prohibits a line break. The command
\verb|\@| in front of a period specifies that this period terminates a
sentence even when it follows an uppercase letter.
\cih{"@} \index{~@ \verb.~.} \index{tilde@tilde ( \verb.~.)}
\index{., space after}

\begin{example}
Mr.~Smith was happy to see her\\
cf.~Fig.~5\\
I like BASIC\@. What about you?
\end{example}

The additional space after periods can be disabled with the command
\begin{lscommand}
\ci{frenchspacing}
\end{lscommand}
\noindent which tells \LaTeX{} \emph{not} to insert more space after a
period than after ordinary character. This is very common in
non-English languages, except bibliographies. If you use
\ci{frenchspacing}, the command \verb|\@| is not necessary.

\section{Titles, Chapters, and Sections}

To help the reader find his or her way through your work, you should
divide it into chapters, sections, and subsections.  \LaTeX{} supports
this with special commands that take the section title as their
argument.  It is up to you to use them in the correct order.

The following sectioning commands are available for the
\texttt{article} class: \nopagebreak

\begin{lscommand}
\ci{section}\verb|{...}|\\
\ci{subsection}\verb|{...}|\\
\ci{subsubsection}\verb|{...}|\\
\ci{paragraph}\verb|{...}|\\
\ci{subparagraph}\verb|{...}|
\end{lscommand}

If you want to split your document in parts without influencing the
section or chapter numbering use
\begin{lscommand}
\ci{part}\verb|{...}|
\end{lscommand}

When you work with the \texttt{report} or \texttt{book} class,
an additional top-level sectioning command becomes available
\begin{lscommand}
\ci{chapter}\verb|{...}|
\end{lscommand}

As the \texttt{article} class does not know about chapters, it is quite easy
to add articles as chapters to a book.
The spacing between sections, the numbering and the font size of the
titles will be set automatically by \LaTeX. 

Two of the sectioning commands are a bit special: 
\begin{itemize}
\item The \ci{part} command does
  not influence the numbering sequence of chapters.  
\item The \ci{appendix} command does not take an argument. It just
  changes the chapter numbering to letters.\footnote{For the article
    style it changes the section numbering.}
\end{itemize}



\LaTeX{} creates a table of contents by taking the section headings
and page numbers from the last compile cycle of the document. The command 
\begin{lscommand} 
\ci{tableofcontents}
\end{lscommand} 
\noindent expands to a table of contents at the place it
is issued. A new
document has to be compiled (``\LaTeX ed'') twice to get a
correct \wi{table of contents}. Sometimes it might be
necessary to compile the document a third time. \LaTeX{} will tell you
when this is necessary.

All sectioning commands listed above also exist as ``starred''
versions.  A ``starred'' version of a command is built by adding a
star \verb|*| after the command name.  This generates section headings
that do not show up in the table of contents and are not
numbered. The command \verb|\section{Help}|, for example, would become
\verb|\section*{Help}|.

Normally the section headings show up in the table of contents exactly
as they are entered in the text. Sometimes this is not possible,
because the heading is too long to fit into the table of contents. The
entry for the table of contents can then be specified as an
optional argument in front of the actual heading.

\begin{code}
\verb|\chapter[Title for the table of contents]{A long|\\
\verb|    and especially boring title, shown in the text}|
\end{code} 

The \wi{title} of the whole document is generated by issuing a 
\begin{lscommand}
\ci{maketitle}
\end{lscommand}
\noindent command. The contents of the title have to be defined by the commands
\begin{lscommand}
\ci{title}\verb|{...}|, \ci{author}\verb|{...}| 
and optionally \ci{date}\verb|{...}| 
\end{lscommand}
\noindent before calling \verb|\maketitle|. In the argument to \ci{author}, you can supply several names separated by \ci{and} commands. 

An example of some of the commands mentioned above can be found in
Figure~\ref{document} on page~\pageref{document}.

Apart from the sectioning commands explained above, \LaTeXe{}
introduced three additional commands for use with the \verb|book| class. 
They are useful for dividing your publication. The commands alter
chapter headings and page numbering to work as you would expect it in
a book:
\begin{description}
\item[\ci{frontmatter}] should be the very first command after
  the start of the document body (\verb|\begin{document}|). It will switch page numbering to Roman
    numerals and sections be non-enumerated. As if you were using 
    the starred sectioning commands (eg \verb|\chapter*{Preface}|)
    but the sections will still show up in the table of contents.
\item[\ci{mainmatter}] comes right before the first chapter of
  the book. It turns on Arabic page numbering and restarts the page
  counter.
\item[\ci{appendix}] marks the start of additional material in your
  book. After this command chapters will be numbered with letters.
\item[\ci{backmatter}] should be inserted before the very last items
  in your book, such as the bibliography and the index. In the standard
  document classes, this has no visual effect.
\end{description}


\section{Cross References}

In books, reports and articles, there are often 
\wi{cross-references} to figures, tables and special segments of text.
\LaTeX{} provides the following commands for cross referencing
\begin{lscommand}
\ci{label}\verb|{|\emph{marker}\verb|}|, \ci{ref}\verb|{|\emph{marker}\verb|}| 
and \ci{pageref}\verb|{|\emph{marker}\verb|}|
\end{lscommand}
\noindent where \emph{marker} is an identifier chosen by the user. \LaTeX{}
replaces \verb|\ref| by the number of the section, subsection, figure,
table, or theorem after which the corresponding \verb|\label| command
was issued. \verb|\pageref| prints the page number of the
page where the \verb|\label| command occurred.\footnote{Note that these commands
  are not aware of what they refer to. \ci{label} just saves the last
  automatically generated number.} As with the section titles, the
numbers from the previous run are used.

\begin{example}
A reference to this subsection
\label{sec:this} looks like:
``see section~\ref{sec:this} on 
page~\pageref{sec:this}.''
\end{example}
 
\section{Footnotes}
With the command
\begin{lscommand}
\ci{footnote}\verb|{|\emph{footnote text}\verb|}|
\end{lscommand}
\noindent a footnote is printed at the foot of the current page.  Footnotes
should always be put\footnote{``put'' is one of the most common
  English words.} after the word or sentence they refer to. Footnotes
referring to a sentence or part of it should therefore be put after
the comma or period.\footnote{Note that footnotes
  distract the reader from the main body of your document. After all,
  everybody reads the footnotes---we are a curious species, so why not
  just integrate everything you want to say into the body of the
  document?\footnotemark}
\footnotetext{A guidepost doesn't necessarily go where it's pointing to :-).}

\begin{example}
Footnotes\footnote{This is 
  a footnote.} are often used 
by people using \LaTeX.
\end{example}
 
\section{Emphasized Words}

If a text is typed using a typewriter, important words are
  \texttt{emphasized by \underline{underlining} them.}
\begin{lscommand}
\ci{underline}\verb|{|\emph{text}\verb|}|
\end{lscommand}
In printed books,
however, words are emphasized by typesetting them in an \emph{italic}
font.  \LaTeX{} provides the command
\begin{lscommand}
\ci{emph}\verb|{|\emph{text}\verb|}|
\end{lscommand}
\noindent to emphasize text.  What the command actually does with 
its argument depends on the context:

\begin{example}
\emph{If you use 
  emphasizing inside a piece
  of emphasized text, then 
  \LaTeX{} uses the
  \emph{normal} font for 
  emphasizing.}
\end{example}

Please note the difference between telling \LaTeX{} to
\emph{emphasize} something and telling it to use a different
\emph{font}:

\begin{example}
\textit{You can also
  \emph{emphasize} text if 
  it is set in italics,} 
\textsf{in a 
  \emph{sans-serif} font,}
\texttt{or in 
  \emph{typewriter} style.}
\end{example}

\section{Environments} \label{env}

% To typeset special purpose text, \LaTeX{} defines many different
% \wi{environment}s for all sorts of formatting:
\begin{lscommand}
\ci{begin}\verb|{|\emph{environment}\verb|}|\quad
   \emph{text}\quad
\ci{end}\verb|{|\emph{environment}\verb|}|
\end{lscommand}
\noindent Where \emph{environment} is the name of the environment. Environments can be
nested within each other as long as the correct nesting order is
maintained.
\begin{code}
\verb|\begin{aaa}...\begin{bbb}...\end{bbb}...\end{aaa}|
\end{code}

\noindent In the following sections all important environments are explained.

\subsection{Itemize, Enumerate, and Description}

The \ei{itemize} environment is suitable for simple lists, the
\ei{enumerate} environment for enumerated lists, and the
\ei{description} environment for descriptions.
\cih{item}

\begin{example}
\flushleft
\begin{enumerate}
\item You can mix the list
environments to your taste:
\begin{itemize}
\item But it might start to
look silly. 
\item[-] With a dash.
\end{itemize}
\item Therefore remember:
\begin{description}
\item[Stupid] things will not
become smart because they are
in a list.
\item[Smart] things, though,
can be presented beautifully
in a list.
\end{description}
\end{enumerate}
\end{example}
 
\subsection{Flushleft, Flushright, and Center}

The environments \ei{flushleft} and \ei{flushright} generate
paragraphs that are either left- or \wi{right-aligned}. \index{left
  aligned} The \ei{center} environment generates centred text. If you
do not issue \ci{\bs} to specify line breaks, \LaTeX{} will
automatically determine line breaks.

\begin{example}
\begin{flushleft}
This text is\\ left-aligned. 
\LaTeX{} is not trying to make 
each line the same length.
\end{flushleft}
\end{example}

\begin{example}
\begin{flushright}
This text is right-\\aligned. 
\LaTeX{} is not trying to make
each line the same length.
\end{flushright}
\end{example}

\begin{example}
\begin{center}
At the centre\\of the earth
\end{center}
\end{example}

\subsection{Quote, Quotation, and Verse}

The \ei{quote} environment is useful for quotes, important phrases and
examples.

\begin{example}
A typographical rule of thumb
for the line length is:
\begin{quote}
On average, no line should
be longer than 66 characters.
\end{quote}
This is why \LaTeX{} pages have 
such large borders by default
and also why multicolumn print
is used in newspapers.
\end{example}

There are two similar environments: the \ei{quotation} and the
\ei{verse} environments. The \texttt{quotation} environment is useful
for longer quotes going over several paragraphs, because it indents the
first line of each paragraph. The \texttt{verse} environment is useful for poems
where the line breaks are important. The lines are separated by
issuing a \ci{\bs} at the end of a line and an empty line after each
verse.


\begin{example}
I know only one English poem by 
heart. It is about Humpty Dumpty.
\begin{flushleft}
\begin{verse}
Humpty Dumpty sat on a wall:\\
Humpty Dumpty had a great fall.\\ 
All the King's horses and all
the King's men\\
Couldn't put Humpty together
again.
\end{verse}
\end{flushleft}
\end{example}

\subsection{Abstract}

In scientific publications it is customary to start with an abstract which
gives the reader a quick overview of what to expect. \LaTeX{} provides the
\ei{abstract} environment for this purpose. Normally \ei{abstract} is used
in documents typeset with the article document class.

\newenvironment{abstract}%
        {\begin{center}\begin{small}\begin{minipage}{0.8\textwidth}}%
        {\end{minipage}\end{small}\end{center}}
\begin{example}
\begin{abstract}
The abstract abstract.
\end{abstract}
\end{example}

\subsection{Printing Verbatim}

Text that is enclosed between \verb|\begin{|\ei{verbatim}\verb|}| and
\verb|\end{verbatim}| will be directly printed, as if typed on a
typewriter, with all line breaks and spaces, without any \LaTeX{}
command being executed.

Within a paragraph, similar behavior can be accessed with
\begin{lscommand}
\ci{verb}\verb|+|\emph{text}\verb|+|
\end{lscommand}
\noindent The \verb|+| is just an example of a delimiter character. Use any
character except letters, \verb|*| or space. Many \LaTeX{} examples in this
booklet are typeset with this command.

\begin{example}
The \verb|\ldots| command \ldots

\begin{verbatim}
10 PRINT "HELLO WORLD ";
20 GOTO 10
\end{verbatim}
\end{example}

\begin{example}
\begin{verbatim*}
the starred version of
the      verbatim   
environment emphasizes
the spaces   in the text
\end{verbatim*}
\end{example}

The \ci{verb} command can be used in a similar fashion with a star:

\begin{example}
\verb*|like   this :-) |
\end{example}

The \texttt{verbatim} environment and the \verb|\verb| command may not be used
within parameters of other commands.

 
\subsection{Tabular}

\newcommand{\mfr}[1]{\framebox{\rule{0pt}{0.7em}\texttt{#1}}}

The \ei{tabular} environment can be used to typeset beautiful
\wi{table}s with optional horizontal and vertical lines. \LaTeX{}
determines the width of the columns automatically.

The \emph{table spec} argument of the 
\begin{lscommand}
\verb|\begin{tabular}[|\emph{pos}\verb|]{|\emph{table spec}\verb|}|
\end{lscommand} 
\noindent command defines the format of the table. Use an \mfr{l} for a column of
left-aligned text, \mfr{r} for right-aligned text, and \mfr{c} for
centred text; \mfr{p\{\emph{width}\}} for a column containing justified
text with line breaks, and \mfr{|} for a vertical line.

If the text in a column is too wide for the page, \LaTeX{} won't
automatically wrap it. Using \mfr{p\{\emph{width}\}} you can define
a special type of column which will wrap-around the text as in a normal paragraph.

The \emph{pos} argument specifies the vertical position of the table
relative to the baseline of the surrounding text.  Use either of the
letters \mfr{t}, \mfr{b} and \mfr{c} to specify table
alignment at the top, bottom or center.
 
Within a \texttt{tabular} environment, \texttt{\&} jumps to the next
column, \ci{\bs} starts a new line and \ci{hline} inserts a horizontal
line.  Add partial lines by using the \ci{cline}\texttt{\{}\emph{j}\texttt{-}\emph{i}\texttt{\}},
where j and i are the column numbers the line should extend over.

\index{"|@ \verb."|.}

\begin{example}
\begin{tabular}{|r|l|}
\hline
7C0 & hexadecimal \\
3700 & octal \\ \cline{2-2}
11111000000 & binary \\
\hline \hline
1984 & decimal \\
\hline
\end{tabular}
\end{example}

\begin{example}
\begin{tabular}{|p{4.7cm}|}
\hline
Welcome to Boxy's paragraph.
We sincerely hope you'll 
all enjoy the show.\\
\hline 
\end{tabular}
\end{example}

The column separator can be specified with the \mfr{@\{...\}}
construct. This command kills the inter-column space and replaces it
with whatever is between the curly braces.  One common use for
this command is explained below in the decimal alignment problem.
Another possible application is to suppress leading space in a table with
\mfr{@\{\}}.

\begin{example}
\begin{tabular}{@{} l @{}}
\hline 
no leading space\\
\hline
\end{tabular}
\end{example}

\begin{example}
\begin{tabular}{l}
\hline
leading space left and right\\
\hline
\end{tabular}
\end{example}

%
% This part by Mike Ressler
%

\index{decimal alignment} Since there is no built-in way to align
numeric columns to a decimal point,\footnote{If the `tools' bundle is
  installed on your system, have a look at the \pai{dcolumn} package.}
we can ``cheat'' and do it by using two columns: a right-aligned
integer and a left-aligned fraction. The \verb|@{.}| command in the
\verb|\begin{tabular}| line replaces the normal inter-column spacing with
just a ``.'', giving the appearance of a single,
decimal-point-justified column.  Don't forget to replace the decimal
point in your numbers with a column separator (\verb|&|)! A column label
can be placed above our numeric ``column'' by using the
\ci{multicolumn} command.
 
\begin{example}
\begin{tabular}{c r @{.} l}
Pi expression       &
\multicolumn{2}{c}{Value} \\
\hline
$\pi$               & 3&1416  \\
$\pi^{\pi}$         & 36&46   \\
$(\pi^{\pi})^{\pi}$ & 80662&7 \\
\end{tabular}
\end{example}

\begin{example}
\begin{tabular}{|c|c|}
\hline
\multicolumn{2}{|c|}{Ene} \\
\hline
Mene & Muh! \\
\hline
\end{tabular}
\end{example}

Material typeset with the tabular environment always stays together on one
page. If you want to typeset long tables, you might want to use the
\pai{longtable} environments.

Sometimes the default \LaTeX{} tables do feel a bit cramped. So you may want
to give them a bit more breathing space by setting a higher
\ci{arraystretch} and \ci{tabcolsep} value.

\begin{example}
\begin{tabular}{|l|}
\hline
These lines\\\hline
are tight\\\hline
\end{tabular}

{\renewcommand{\arraystretch}{1.5}
\renewcommand{\tabcolsep}{0.2cm}
\begin{tabular}{|l|}
\hline
less cramped\\\hline
table layout\\\hline
\end{tabular}}

\end{example}

If you just want to grow the height of a single row in your table add an invisible vertical bar\footnote{In professional typesetting,
this is called a \wi{strut}.} Use a zero width \ci{rule} to implement this trick.

\begin{example}
\begin{tabular}{|c|}
\hline
\rule{1pt}{4ex}Pitprop \ldots\\
\hline
\rule{0pt}{4ex}Strut\\
\hline
\end{tabular}
\end{example}

\section{Floating Bodies}
Today most publications contain a lot of figures and tables. These
elements need special treatment, because they cannot be broken across
pages.  One method would be to start a new page every time a figure or
a table is too large to fit on the present page. This approach would
leave pages partially empty, which looks very bad.

The solution to this problem is to `float' any figure or table that
does not fit on the current page to a later page, while filling the
current page with body text. \LaTeX{} offers two environments for
\wi{floating bodies}; one for tables and  one for figures.  To
take full advantage of these two environments it is important to
understand approximately how \LaTeX{} handles floats internally.
Otherwise floats may become a major source of frustration, because
\LaTeX{} never puts them where you want them to be.

\bigskip
Let's first have a look at the commands \LaTeX{} supplies
for floats:

Any material enclosed in a \ei{figure} or \ei{table} environment will
be treated as floating matter. Both float environments support an optional
parameter
\begin{lscommand}
\verb|\begin{figure}[|\emph{placement specifier}\verb|]| or
\verb|\begin{table}[|\ldots\verb|]|
\end{lscommand}
\noindent called the \emph{placement specifier}. This parameter
is used to tell \LaTeX{} about the locations to which the float
is allowed to be moved.  A \emph{placement specifier} is constructed by building a string
of \emph{float-placing permissions}. See Table~\ref{tab:permiss}.

\begin{table}[!bp]
\caption{Float Placing Permissions.}\label{tab:permiss}
\noindent \begin{minipage}{\textwidth}
\medskip
\begin{center}
\begin{tabular}{@{}cp{8cm}@{}}
Spec&Permission to place the float \ldots\\
\hline
\rule{0pt}{1.05em}\texttt{h} & \emph{here} at the very place in the text
  where it occurred.  This is useful mainly for small floats.\\[0.3ex]
\texttt{t} & at the \emph{top} of a page\\[0.3ex]
\texttt{b} & at the \emph{bottom} of a page\\[0.3ex]
\texttt{p} & on a special \emph{page} containing only floats.\\[0.3ex]
\texttt{!} & without considering most of the  internal parameters\footnote{Such as the
    maximum number of floats allowed  on one page.}, which could stop this
  float from being placed.
\end{tabular}
\end{center}
Note that \texttt{pt} and \texttt{em} are \TeX{} units. Read more
on this in table \ref{units} on page \pageref{units}.
\end{minipage}
\end{table}

A table could be started with the following line e.g.\
\begin{code}
\verb|\begin{table}[!hbp]|
\end{code}
\noindent The \wi{placement specifier} \verb|[!hbp]| allows \LaTeX{} to 
place the table right here (\texttt{h}) or at the bottom (\texttt{b}) 
of some page
or on a special floats page (\texttt{p}), and all this even if it does not
look that good (\texttt{!}). If no placement specifier is given, the standard
classes assume \verb|[tbp]|.

\LaTeX{} will place every float it encounters according to the
placement specifier supplied by the author. If a float cannot be
placed on the current page it is deferred either to the
\emph{figures} or the \emph{tables} queue.\footnote{These are FIFO---`first in first out'---queues!}  When a new page is started,
\LaTeX{} first checks if it is possible to fill a special `float'
page with floats from the queues. If this is not possible, the first
float on each queue is treated as if it had just occurred in the
text: \LaTeX{} tries again to place it according to its
respective placement specifiers (except `h,' which is no longer
possible).  Any new floats occurring in the text get placed into the
appropriate queues. \LaTeX{} strictly maintains the original order of
appearance for each type of float. That's why a figure that cannot
be placed pushes all further figures to the end of the document.
Therefore:

\begin{quote}
If \LaTeX{} is not placing the floats as you expected,
it is often only one float jamming one of the two float queues.
\end{quote}                 

While it is possible to give \LaTeX{}  single-location placement
specifiers, this causes problems.  If the float does not fit in the
location specified it becomes stuck, blocking subsequent floats.
In particular, you should never, ever use the [h] option---it is so bad
that in more recent versions of \LaTeX, it is automatically replaced by
[ht].

\bigskip
\noindent Having explained the difficult bit, there are some more things to
mention about the \ei{table} and \ei{figure} environments.
Use the 

\begin{lscommand}
\ci{caption}\verb|{|\emph{caption text}\verb|}|
\end{lscommand}

\noindent command to define a caption for the float. A running number and
the string ``Figure'' or ``Table'' will be added by \LaTeX.

The two commands

\begin{lscommand}
\ci{listoffigures} and \ci{listoftables} 
\end{lscommand}

\noindent operate analogously to the \verb|\tableofcontents| command,
printing a list of figures or tables, respectively.  These lists will
display the whole caption, so if you tend to use long captions
you must have a shorter version of the caption for the lists.
This is accomplished by entering the short version in brackets after
the \verb|\caption| command.
\begin{code}
\verb|\caption[Short]{LLLLLoooooonnnnnggggg}| 
\end{code}

Use \ci{label} and \ci{ref},to create a reference to a float within
your text. Note that the \ci{label} command must come \emph{after} the
\ci{caption} command since you want it to reference the number of the
caption.

The following example draws a square and inserts it into the
document. You could use this if you wanted to reserve space for images
you are going to paste into the finished document.

\begin{code}
\begin{verbatim}
Figure~\ref{white} is an example of Pop-Art.
\begin{figure}[!hbtp]
\makebox[\textwidth]{\framebox[5cm]{\rule{0pt}{5cm}}}
\caption{Five by Five in Centimetres.\label{white}}
\end{figure}
\end{verbatim}
\end{code}

\noindent In the example above, 
\LaTeX{} will try \emph{really hard}~(\texttt{!})\ to place the figure
right \emph{here}~(\texttt{h}).\footnote{assuming the figure queue is
  empty.} If this is not possible, it tries to place the figure at the
\emph{bottom}~(\texttt{b}) of the page.  Failing to place the figure
on the current page, it determines whether it is possible to create a float
page containing this figure and maybe some tables from the tables
queue. If there is not enough material for a special float page,
\LaTeX{} starts a new page, and once more treats the figure as if it
had just occurred in the text.

Under certain circumstances it might be necessary to use the 

\begin{lscommand}
\ci{clearpage} or even the \ci{cleardoublepage} 
\end{lscommand}

\noindent command. It orders \LaTeX{} to immediately place all 
floats remaining in the queues and then start a new
page. \ci{cleardoublepage} even goes to a new right-hand page.

You will learn how to include \PSi{}
drawings into your \LaTeXe{} documents later in this introduction.

\section{Protecting Fragile Commands}

Text given as arguments of commands like \ci{caption} or \ci{section} may
show up more than once in the document (e.g.\ in the table of contents as
well as in the body of the document). Some commands will break when used in
the argument of \ci{section}-like commands. Compilation of your document
will fail. These commands are called \wi{fragile commands}---for example,
\ci{footnote} or \ci{phantom}. These fragile commands need protection (don't
we all?). Protect them by putting the \ci{protect} command in front
of them.

\ci{protect} only refers to the command that follows right behind, not even
to its arguments. In most cases a superfluous \ci{protect} won't hurt.

\begin{code}
\verb|\section{I am considerate|\\
\verb|      \protect\footnote{and protect my footnotes}}|
\end{code}

% Local Variables:
% TeX-master: "lshort2e"
% mode: latex
% mode: flyspell
% End:
